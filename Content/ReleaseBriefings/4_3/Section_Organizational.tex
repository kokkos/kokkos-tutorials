
%==========================================================================

\begin{frame}[fragile]

  {\Huge Organizational}

  \vspace{10pt}

  \textbf{Content:}
  \begin{itemize}
    \item Linux Foundation
    \item{\texttt{master} Branch}
    \item Kokkos Tea Time
  \end{itemize}

\end{frame}

%==========================================================================

\begin{frame}[fragile]{Linux Foundation}
\begin{center}
\textbf{Kokkos is now a project of the Linux Foundation!}
\end{center}

At ISC HPC in Hamburg (May 2024) we will also join the \textit{High Performance Software Foundation}!


\textit{Why did you do that?} To grow developer base!
\begin{itemize}
\item{LF provides neutral ground}
\item{Well defined governance}
\item{Not a "DOE" or even worse "Sandia" project}
\end{itemize}

\textit{What changes for users?} Mostly nothing!
\begin{itemize}
\item{Hopefully even better Kokkos in the long run!}
\item{Will leverage LF to organize community events.}
\item{Only caveat: Trademark rules}
\end{itemize}
\end{frame}

\begin{frame}[fragile]{Linux Foundation - Trademark}
\textit{Trademark???} Yes: Kokkos\texttrademark ...
\begin{itemize}
\item Don't call your project "KokkosFoo" unless inside the Kokkos organization
\item You can say "Foo based on Kokkos" "Foo for Kokkos" and disclaim "Kokkos is a LF project"
\item You can organize classes, presentations and what not on Kokkos, and refer to it as much as you like in presentations.
\item The goal is to avoid confusion of what is part of the official Kokkos LF project vs efforts which are just leveraging the project.
\end{itemize}

If you have questions: asks us!

And look at the Linux Foundation Trademark rules: \url{https://www.linuxfoundation.org/legal/trademark-usage}
\end{frame}

\begin{frame}[fragile]{\texttt{master} Branch}
\textbf{We are getting rid of the master branch for Kokkos}

\begin{itemize}
\item{\texttt{master} branch has been the "latest release"}
\item{Not generally a common practice on github e.g. normally \texttt{main} is what we call \texttt{develop}}
\item{Often lead to PRs based on wrong thing}
\end{itemize}

\textit{What will change?}

\begin{itemize}
\item{\texttt{develop} branch now the default branch}
\item{\texttt{master} will still exist for a little while}
\end{itemize}

\end{frame}

\begin{frame}[fragile]{Kokkos Tea Time}
\emph{Kokkos tea-time} is a monthly time online to meet the community and discuss anything related to \textbf{Kokkos}, its \textbf{ecosystem}, or even \textbf{GPU-programming} at large.
\medskip
\begin{itemize}
  \item 3rd Wednesday of the month
  \begin{itemize}
   \item {\small7AM PT, 8AM MT, 10AM ET, 2PM UTC, 4PM CEST}
  \end{itemize}
  \item a 45min time slot for either
  \begin{itemize}
    \item a 30min presentation followed by questions,
    \item or a more informal discussion on a select topic.
  \end{itemize}
  \item by zoom, by visio, by phone, or in your nearest\\ kokkos shop
  \begin{itemize}
   \item Get you link \url{https://cexa-project.org/news/}
  \end{itemize}
\end{itemize}
\vfill
Discover Kokkos Resilience with Nic Morales\\
on the 16$^{\text{th}}$ of May.

\begin{textblock}{4.5}(10.25,5.75)
\centering
\includegraphics[width=.45\textwidth]{4_3/tea-time_we_want_you}

We want you to tell us about your Kokkos use, development, ideas, ...\\
{\small\url{contact@cexa-project.org}}
\end{textblock}
\begin{textblock}{1.5}(13.75,4.25)
\includegraphics[width=\textwidth]{4_3/tea-time-QR}
\end{textblock}
\end{frame}

%==========================================================================

\begin{frame}[fragile]{New Kokkos Efforts}
\textit{Two new Kokkos subprojects:}

\begin{itemize}
\item Very Experimental: expect changes of interface etc.
\item Brave early experimenters welcome to help give feedback or get involved.
\end{itemize}

\textbf{KokkosFFT}
\begin{itemize}
\item Goal: wrapping existing MPI libraries such as fftw, cufft, mkl and rocfft.
\item \url{https://github.com/kokkos/kokkos-fft}
\item POC: Yuuichi Asahi (CEA)
\end{itemize}

\textbf{KokkosComm}
\begin{itemize}
\item Goal: provide communication facilities: for now MPI-like interfaces taking \texttt{View}s.
\item \url{https://github.com/kokkos/kokkos-comm}
\item Join the \texttt{mpi-interop} channel on the Kokkos Slack
\end{itemize}

\end{frame}
