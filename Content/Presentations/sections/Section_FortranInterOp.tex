
% Motivation
%  - data interop with fortran arrays
%  - have arrays in fortran alias Kokkos::Views allocation
% 1) Application in Fortran, performance critical stuff in C++
%  - how to create views/arrays and alias them
% 2) Application in C++ some physics modules in Fortran
% 3) Calling serial CUDA Fortran functions from a kernel
%  - in the future maybe fortran OpenMP target functions
\begin{frame}[fragile]

  {\Huge Fortran InterOp}

  \vspace{10pt}

  {\large How to write hybrid Fortran - Kokkos code.}

  \vspace{20pt}

  \textbf{Learning objectives:}
  \begin{itemize}
    \item {Allocating data in Fortran and viewing it as Kokkos Views.}
    \item {Calling C++ functions with Kokkos in it from Fortran.}
    \item {Allocating DualView's from within Fortran.}
  \end{itemize}

  \vspace{-20pt}

\end{frame}

%==========================================================================

\begin{frame}[fragile]{Why do we need this?}

\textbf{HPC world owns many Fortran LOC!}

\vspace{10pt}
\begin{itemize}
  \item We generally cannot port it all at once.
  \item We need an incremental porting strategy
  \begin{itemize}
    \item Keep our e.g. Fortran mains, drivers, physics packages
    \item But port relevant infrastructure, or hotspot kernels to C++ and Kokkos
  \end{itemize}
\end{itemize}

\vspace{10pt}
\pause
\textbf{How do we make Kokkos and Fortran talk with each other?}
\end{frame}

\begin{frame}[fragile]{FLCL}
\textbf{Fortran Language Compatibility Layer (FLCL)}

\begin{itemize}
  \item Pass multidimensional arrays accross the C++/Fortran boundary
  \begin{itemize}
    \item See Fortran arrays as Kokkos Views and vice versa
  \end{itemize}
  \item Create Kokkos View and DualView from Fortran
  \begin{itemize}
     \item Allows Fortran to be the memory owner but call C++ functions with Kokkos kernels for CUDA/HIP
  \end{itemize}
  \item Initialize and Finalize Kokkos from Fortran
  \item FortranIndex$<$T$>$ scalar type to deal with 1 vs 0 based indexing in sparse data structures
\end{itemize}

\pause
\begin{block}{FLCL}
  The Fortran Language Compatibility Layer allows an incremental porting of a Fortran code to Kokkos.
\end{block}
\end{frame}

\begin{frame}[fragile]{Initializing Kokkos}
\textbf{Simple binding of \texttt{Kokkos::initialize} and \texttt{Kokkos::finalize}}

\vspace{5pt}
\begin{itemize}
  \item \texttt{kokkos\_initialize()}
  \begin{itemize}
    \item Call after \texttt{MPI\_Initialize}
    \item Parses the command line arguments of the executable
  \end{itemize}
  \item \texttt{kokkos\_initialize\_without\_args()}
  \begin{itemize}
    \item Call after \texttt{MPI\_Initialize}
    \item Ignores command line arguments of the executable
    \item Kokkos will still look up environment variables
  \end{itemize}
  \item \texttt{kokkos\_finalize}
  \begin{itemize}
    \item Call before \texttt{MPI\_Finalize}
  \end{itemize}
\end{itemize}
\end{frame}

\begin{frame}[fragile]{nd\_array\_t}
\begin{block}{nd\_array\_t}
  The compatibility glue between Fortran arrays and Kokkos Views.
\end{block}

\pause
\textbf{Keeps Track of:}
\begin{itemize}
  \item An array's rank
  \item Extents of the array
  \item Strides of the array
  \item Pointer to the allocation
\end{itemize}

\pause
\vspace{5pt}
\textbf{How do we create an \texttt{nd\_array\_t}?}
\begin{itemize}
  \item Explicit routines like \texttt{to\_nd\_array\_i64\_d6}
  \item Simple interface taking a fortran array as argument
  \begin{code}
    array = to_nd_array(foo)  ! Fortran
  \end{code}
\end{itemize}

\pause
\textbf{This allows us to write a simple hybrid Fortran/Kokkos code!}
\end{frame}

\begin{frame}[fragile]{AXPBY}
\textbf{Everyone loves AXPBY so we do it here!}

\begin{itemize}
  \item Using a few modules including iso\_c\_binding and the flcl\_mod provided by FLCL
  \item \texttt{axpby} is just a fortran subroutine taking fortran arguments (next slide)
\end{itemize}

\begin{code}[keywords={program,axpy,flcl_mod,axpy_f_mod,kokkos_initialize,kokkos_finalize}]
program example_axpy
  use, intrinsic :: iso_c_binding
  use :: flcl_mod
  use :: axpy_f_mod
  implicit none
 
  real(c_double), dimension(:), allocatable :: c_y
  real(c_double), dimension(:), allocatable :: x
  real(c_double) :: alpha
  integer :: mm = 5000
  ... setup here ...
  call kokkos_initialize()
  call axpy(c_y, x, alpha)
  call kokkos_finalize()
end program example_axpy
\end{code}


\end{frame}

\begin{frame}[fragile]{AXPBY}
\begin{code}[keywords={axpy,f_axpy,c_axpy,iso_c_binding,flcl_mod,to_nd_array,nd_array_y,nd_array_x,nd_array_t}]
module axpy_f_mod
    use, intrinsic :: iso_c_binding
    use :: flcl_mod
    public
      interface
        subroutine f_axpy( nd_array_y, nd_array_x, alpha ) &
          & bind(c, name='c_axpy')
          use, intrinsic :: iso_c_binding
          use :: flcl_mod
          type(nd_array_t) :: nd_array_y
          type(nd_array_t) :: nd_array_x
          real(c_double) :: alpha
        end subroutine f_axpy
      end interface
      contains
        subroutine axpy( y, x, alpha )
          use, intrinsic :: iso_c_binding
          use :: flcl_mod
          implicit none
          real(c_double), dimension(:), intent(inout) :: y
          real(c_double), dimension(:), intent(in) :: x
          real(c_double), intent(in) :: alpha
          call f_axpy(to_nd_array(y), to_nd_array(x), alpha)
        end subroutine axpy
end module axpy_f_mod
\end{code}
\end{frame}

\begin{frame}[fragile]{AXPY - The C++}

In C++ create an unmanaged view from the \texttt{nd\_array\_t} handle:

\begin{code}[keywords={flcl,extern,flcl_ndarray_t,view_from_ndarray}]
#include "flcl-cxx.hpp"
extern "C" {
  void c_axpy( flcl_ndarray_t *nd_array_y,
               flcl_ndarray_t *nd_array_x,
               double *alpha )
  {
    auto y = flcl::view_from_ndarray<double*>(*nd_array_y);
    auto x = flcl::view_from_ndarray<double*>(*nd_array_x);

    Kokkos::parallel_for( "axpy", y.extent(0), 
      KOKKOS_LAMBDA( const size_t idx) {
      y(idx) += *alpha * x(idx);
    });
  }
}
\end{code}

\begin{itemize}
  \item All the arrays are \texttt{LayoutLeft} i.e. Fortran Layout
  \item The data type needs to match.
\end{itemize}
\end{frame}

\begin{frame}[fragile]{DualView Allocation}
\textbf{FLCL allows allocating DualViews from Fortran}
\begin{code}[keywords={kokkos_allocate_dualview,c_ptr}]
real(c_double), dimension(:), pointer :: array_x
type(c_ptr) :: v_x
... setup here ...
call kokkos_allocate_dualview(array_x, v_x, "array_x", length)
\end{code}

\begin{itemize}
  \item \texttt{array\_x} is an array aliasing the host view of the dualview
  \item \texttt{v\_x} is a pointer to the DualView itself.
\end{itemize}

\pause
In C++ take a pointer to a pointer of a DualView as argument:

\begin{code}[keywords={flcl,dualview_r64_1d_t}]
#include "flcl-cxx.hpp"
void c_foo( flcl::dualview_r64_1d_t**  v_x ) {
  flcl::dualview_r64_1d_t dv_x = **v_x;
  dv_x.sync_device();
  ...
}
\end{code}

\vspace{-4pt}
\begin{itemize}
  \item Can assign to an instance, DualView is reference counted
  \item Note: there is NO type safety in the Fortran/C++ boundary, better make sure you get the types right!
\end{itemize}

\end{frame}

\begin{frame}[fragile]{DualView GlueCode}

\textbf{Write the same gluecode as for AXPY example:}

\begin{code}[keywords={foo,c_foo,flcl_mod,flcl,dualview_r64_1d_t,c_ptr}]
module interface_f_mod
  use, intrinsic :: iso_c_binding
  use :: flcl_mod
  implicit none
  public
  interface
    subroutine foo( v_x ) &
      & bind(c, name="c_foo")
      use, intrinsic :: iso_c_binding
      use :: flcl_mod
      implicit none
      type(c_ptr), intent(inout) :: v_x
    end subroutine foo
  end interface
end module interface_f_mod

void c_foo( flcl::dualview_r64_1d_t**  v_x ) {
  flcl::dualview_r64_1d_t dv_x = **v_x;
  dv_x.sync_device();
  ...
}
\end{code}
\end{frame}

\begin{frame}[fragile]{Summary}
\begin{itemize}
  \item Fortran Language Compatibility Layer provides facilities for interoperability of Kokkos and Fortran
  \item Initialize Kokkos from Fortran via \texttt{kokkos\_initialize} and \texttt{kokkos\_finalize}
  \item \texttt{nd\_array\_t} is a representation of a \texttt{Kokkos::View}
  \item Create \texttt{nd\_array\_t} from a Fortran array via \texttt{to\_nd\_array}
  \item Allocate \texttt{Kokkos::DualView} in Fortran with \texttt{kokkos\_allocate\_dualview}
\end{itemize}

\vspace{10pt}
\textbf{Available at \url{https://github.com/kokkos/kokkos-fortran-interop}.}
\begin{itemize}
  \item Feedback is appreciated!
\end{itemize}
\end{frame}
