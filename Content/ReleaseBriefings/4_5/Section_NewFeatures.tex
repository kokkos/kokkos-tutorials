%==========================================================================

\begin{frame}[fragile]

  {\Huge Feature highlights}

  \vspace{10pt}

\end{frame}

%==========================================================================

% Examples

% note: always keep the [fragile] for your frames!

%\begin{frame}[fragile]{Example list}
%  \begin{itemize}
%      \item Item 1
%      \item Item 2 with some \texttt{code}
%      \begin{itemize}
%        \item Sub-item 2.1
%        \item Sub-item 2.2
%      \end{itemize}
%  \end{itemize}
%\end{frame}

%\begin{frame}[fragile]{Example code}
%    \begin{code}[keywords={std}]
%        #include <iostream>
%        
%        int main() {
%            std::cout << "hello world\n";
%        }
%    \end{code}
%\end{frame}

%\begin{frame}[fragile]{Example table}
%    \begin{center}
%        \begin{tabular}{l|l}
%            a & b \\\hline
%            c & d
%        \end{tabular}
%    \end{center}
%\end{frame}

%==========================================================================

\begin{frame}[fragile]{Support building without RTTI}
Kokkos can now be built with Run-Time Type Information (RTTI) disabled.
\vspace{10pt}
\begin{itemize}
\item RTTI is required to deremine the type of an object during program execution.
\item It is used by:
  \begin{itemize}
  \item Exception handling
  \item \texttt{dynamic\_cast}
  \item \texttt{typeid} operator and \texttt{std::type\_info} class
  \end{itemize}
\item It can be disabled at configuration via \texttt{-DCMAKE\_CXX\_FLAGS="-fno-rtti"}
\item Typical use case is to reduce the binary size when targeting systems with limited amount of memory
\end{itemize}
\end{frame}

%==========================================================================

\begin{frame}[fragile]{New \texttt{KOKKOS\_RELOCATABLE\_FUNCTION} annotation macro}
Required when using Relocatable Device Code (RDC) with SYCL
\begin{itemize}
%\item Also known as separable compilation mode
\item \texttt{KOKKOS\_RELOCATABLE\_FUNCTION} function annotation macro expands to
  \begin{itemize}
  \item \texttt{extern \_\_host\_\_ \_\_device\_\_} with CUDA or HIP
  \item \texttt{extern SYCL\_EXTERNAL} with SYCL
  \end{itemize}
\item \texttt{-DKokkos\_ENABLE\_\{CUDA,HIP,SYCL\}\_RECLOCATABLE\_DEVICE\_CODE=ON}
\item Can have non-trivial perfomance implications
\end{itemize}

\begin{code}
// foo.cpp
#include <Kokkos_Core.hpp>
KOKKOS_RELOCATABLE_FUNCTION void foo() { Kokkos::printf("foo\n"); }

// bar.cpp
#include <Kokkos_Core.hpp>
KOKKOS_RELOCATABLE_FUNCTION void foo();
void bar() { Kokkos::parallel_for(1, KOKKOS_LAMBDA(int) { foo(); }); }
\end{code}
\end{frame}

%==========================================================================

\begin{frame}[fragile]{SYCL backend has matured}
The SYCL backend is out of the namespace \texttt{Experimental::}.
\begin{itemize}
\item Keep non-deprecated aliases for now but will deprecate in upcoming releases (potentially in 4.6)
\item (Hopefully) very limited impact on user code when leveraging portable aliases (\texttt{DefaultExecutionSpace}, \texttt{SharedSpace}, \texttt{SharedHostPinnedSpace}, etc.)
\item If you specialized something for SYCL and really have to spell the SYCL class names, you can always do
\begin{code}
#if KOKKOS_VERSION_GREATER_EQUAL(4, 5, 0)
using MySyclExec = Kokkos::SYCL;
#else
using MySyclExec = Kokkos::Experimental::SYCL;
#endif
\end{code}
\end{itemize}

\end{frame}
