
% \begin{frame}{DOE ECP Acknowledgement}

% \textit{
% This research was supported by the Exascale Computing Project (17-SC-20-SC),
% a joint project of the U.S. Department of Energy’s Office of Science and National Nuclear Security Administration,
% responsible for delivering a capable exascale ecosystem, including software, applications, and hardware technology,
% to support the nation’s exascale computing imperative. 
% }

% \end{frame}

%==============================================================================

\begin{frame}{Welcome to Kokkos}

\textbf{Kokkos is C++ Performance Portability}
\begin{scriptsize}
  \begin{itemize} 
	  \item {Write a \textit{single source} implementation using C++}
	  \item {Use a \textit{descriptive} Programming Model}
	  \item {Compile for GPUs and CPUs}
  \end{itemize}
\end{scriptsize}

	\vspace{10pt}

\textbf{Kokkos is Ready for Use}
\begin{scriptsize}
  \begin{itemize} 
	  \item {Well established project since 2012}
	  \item {Major buy-in by DOE National Labs}
	  \item {Well over 100 projects with over 500 developers use Kokkos}
	  \item {Dedicated developer staff at 5 National Labs}
	  \item {Robust support for software stacks: GCC 8+, Clang 8+, NVCC 11+, ROCM 5.2, Intel 19+}
  \end{itemize}
\end{scriptsize}

\end{frame}


\begin{frame}{Welcome to Kokkos}

\textbf{Online Resources}:

\begin{itemize}
	\item \url{https://github.com/kokkos}: 
		\begin{itemize}
			\item Primary Kokkos GitHub Organization
		\end{itemize}
	\item \url{https://github.com/kokkos/kokkos-tutorials/LectureSeries}: 
		\begin{itemize}
			\item{Find these slides}
		\end{itemize}
	\item \url{https://github.com/kokkos/kokkos/wiki}: 
		\begin{itemize}
			\item Wiki including API reference
		\end{itemize}
	\item \url{https://kokkosteam.slack.com}: 
		\begin{itemize}
			\item Slack channel for Kokkos.
			\item Please join: fastest way to get your questions answered.
			\item Can whitelist domains, or invite individual people. Email: crtrott@sandia.gov
		\end{itemize}
\end{itemize}

\end{frame}


%==============================================================================

\begin{frame}{Lecture Series Outline}

\begin{itemize}
	\item Module 1: Introduction, Building and Parallel Dispatch
	\item Module 2: Views and Spaces
	\item Module 3: Data Structures + MultiDimensional Loops
	\item Module 4: Hierarchical Parallelism
	\item Module 5: Tasking, Streams and SIMD
	\item Module 6: Internode: MPI and PGAS
	\item Module 7: Tools: Profiling, Tuning and Debugging
	\item Module 8: Kernels: Sparse and Dense Linear Algebra
    \item Module 9: Fortran inter-op
\end{itemize}
\end{frame}

\begin{frame}{What to Expect}

\textbf{Lectures}
\begin{itemize}
	\item Typically 90 minutes of lecture
	\item Submodules have associated exercise as homework
	\item Typically 2-3 Exercises per lecture
	\item Exercises will be talked through at next meeting.
\end{itemize}

	\vspace{10pt}

\textbf{Exercises}
\begin{itemize}
	\item Exercises are small codes with places to do modifications.
	\item Access to GPUs helpful for most of them, but most can be done on pure CPU systems.
	\item Only dependent on standard compilers (e.g. Clang, NVCC)
	\item Office hours on Tuesdays 3-5 PM Eastern Time (potentially with AWS access).
        \item Ongoing support at \url{https://kokkosteam.slack.com}
\end{itemize}
\end{frame}

\begin{frame}{Module 1}
  \begin{block}{Introduction}
    What is Kokkos? Who is behind it? Why should you use it?
  \end{block}

  \begin{block}{Parallel Dispatch}
    Pattern, Policy and Body: how to parallelize simple code with Kokkos.
  \end{block}

  \begin{block}{Building}
    What do you need to build Kokkos and Apps? How to integrate into your build system?
  \end{block}

\end{frame}
