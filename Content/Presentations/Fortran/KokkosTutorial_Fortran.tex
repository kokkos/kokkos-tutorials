
\documentclass[dvipsnames, aspectratio=169]{beamer}
\usepackage[utf8]{inputenc}
\usepackage{listings}
\usepackage{comment}
\usepackage{soul}
%\usepackage{ulem}
\usepackage{subfig}
\usepackage{pgf-pie}
\setul{}{1pt}
\usepackage[oldenum, olditem]{paralist}
%allow even smaller text
\newcommand\tinytiny{\fontsize{4pt}{3}\selectfont}

\makeatletter
\let\old@lstKV@SwitchCases\lstKV@SwitchCases
\def\lstKV@SwitchCases#1#2#3{}
\makeatother
\usepackage{lstlinebgrd}
\makeatletter
\let\lstKV@SwitchCases\old@lstKV@SwitchCases

\lst@Key{numbers}{none}{%
    \def\lst@PlaceNumber{\lst@linebgrd}%
    \lstKV@SwitchCases{#1}%
    {none:\\%
     left:\def\lst@PlaceNumber{\llap{\normalfont
                \lst@numberstyle{\thelstnumber}\kern\lst@numbersep}\lst@linebgrd}\\%
     right:\def\lst@PlaceNumber{\rlap{\normalfont
                \kern\linewidth \kern\lst@numbersep
                \lst@numberstyle{\thelstnumber}}\lst@linebgrd}%
    }{\PackageError{Listings}{Numbers #1 unknown}\@ehc}}
\makeatother



\title{Kokkos-Fortran-Interop}

\usetheme{kokkos}

\newif\ifshort
\newif\ifmedium
\newif\iffull
\newif\ifnotoverview

\newcommand{\TutorialDirectory}{\texttt{Intro-Full}}
\newcommand{\ExerciseDirectory}[1]{\texttt{Exercises/#1/}}
\newcommand{\TutorialClone}{\texttt{Kokkos/kokkos-tutorials/\TutorialDirectory}}

\definecolor{darkgreen}{rgb}{0.0, 0.5, 0.0}
\definecolor{darkred}{rgb}{0.8, 0.0, 0.0}
\definecolor{orange}{rgb}{0.8, 0.33, 0.0}
\definecolor{purple}{rgb}{0.60, 0.20, 0.80}
\colorlet{bodyColor}{blue!20}
\colorlet{patternColor}{orange!30}
\colorlet{policyColor}{green!30}

% http://tex.stackexchange.com/questions/144448/color-a-text-line-in-a-code-lstlisting
\lstnewenvironment{code}[1][]%
{
  %with txfonts: OT1/txr/m/n/10
  %with default fonts: OT1/cmr/m/n/10
  %\fontfamily{cmr}\selectfont
  %\showthe\font
   \noindent
   \minipage{\linewidth}
   %\vspace{0.5\baselineskip}
   \lstset{mathescape, escapeinside={<@}{@>},
moredelim=**[is][{\btHL[fill=patternColor]}]{@pattern}{@pattern},
moredelim=**[is][{\btHL[fill=red!30]}]{@warning}{@warning},
moredelim=**[is][{\btHL[fill=policyColor]}]{@policy}{@policy},
moredelim=**[is][{\btHL[fill=bodyColor]}]{@body}{@body},
moredelim=**[is][{\btHL[fill=red!30]}]{@warning}{@warning},
moredelim=**[is][\color{black}]{@black}{@black},
moredelim=**[is][\color{blue}]{@blue}{@blue},
moredelim=**[is][\bf]{@bold}{@bold},
moredelim=**[is][\it]{@italic}{@italic},
moredelim=**[is][\color{boldblue}\bf]{@boldblue}{@boldblue},
moredelim=**[is][\color{red}]{@red}{@red},
moredelim=**[is][\color{green}]{@green}{@green},
moredelim=**[is][\color{gray}]{@gray}{@gray},
moredelim=**[is][\color{darkgreen}]{@darkgreen}{@darkgreen},
moredelim=**[is][\color{darkred}]{@darkred}{@darkred},
moredelim=**[is][\color{orange}]{@orange}{@orange},
moredelim=**[is][\color{purple}]{@purple}{@purple},
keywords={},
#1}
}
{
  \endminipage
  %\vspace{1.0\baselineskip}
}

\makeatletter
\newif\ifATOlinebackground
\lst@Key{linebackground}{\tiny}{\def\ATOlinebackground{#1}\global\ATOlinebackgroundtrue}
\makeatother

\lstnewenvironment{shell}[1][]{%
  \global\ATOlinebackgroundfalse
  \lstset{language=sh,%
    showstringspaces=false,
    aboveskip=0pt,
    frame=none,
    numbers=none,
    belowskip=2pt,
    breaklines=true,
    #1,
    }
  %\ifATOlinebackground
  \lstset{linebackgroundcolor={
    \ATOlinebackground
  }}
  %\fi
  }{}

\lstnewenvironment{cmake}[1][]{%
  \global\ATOlinebackgroundfalse
  \lstset{language=sh,%
    showstringspaces=false,
    aboveskip=0pt,
    frame=none,
    numbers=none,
    belowskip=2pt,
    breaklines=true,
    #1,
    }
  %\ifATOlinebackground
  \lstset{linebackgroundcolor={
    \ATOlinebackground
  }}
  %\fi
  }{}

\newcommand{\inlinecode}[1]{{\lstset{basicstyle=\ttfamily,keywordstyle={},showstringspaces=false}\lstinline$#1$}}
\newcommand{\inlineshell}[1]{{\lstset{basicstyle=\ttfamily,keywordstyle={},showstringspaces=false}\lstinline$#1$}}

\setbeamercolor{block title}{fg=white, bg=SandiaLightBlue}
\setbeamercolor{block body}{bg=lightgray}
\setbeamercolor{block title alerted}{fg=white, bg=SandiaRed}
\setbeamercolor{block body alerted}{bg=lightgray}



%\usepackage[texcoord,grid,gridunit=mm,gridcolor=red!10,subgridcolor=green!10]{eso-pic}
\usepackage[absolute,overlay]{textpos}





% http://tex.stackexchange.com/questions/8851/how-can-i-highlight-some-lines-from-source-code

\usepackage{pgf, pgffor}
\usepackage{listings}
\usepackage{lstlinebgrd} % see http://www.ctan.org/pkg/lstaddons

\makeatletter
%%%%%%%%%%%%%%%%%%%%%%%%%%%%%%%%%%%%%%%%%%%%%%%%%%%%%%%%%%%%%%%%%%%%%%%%%%%%%%
%
% \btIfInRange{number}{range list}{TRUE}{FALSE}
%
% Test in int number <number> is element of a (comma separated) list of ranges
% (such as: {1,3-5,7,10-12,14}) and processes <TRUE> or <FALSE> respectively

\newcount\bt@rangea
\newcount\bt@rangeb

\newcommand\btIfInRange[2]{%
    \global\let\bt@inrange\@secondoftwo%
    \edef\bt@rangelist{#2}%
    \foreach \range in \bt@rangelist {%
        \afterassignment\bt@getrangeb%
        \bt@rangea=0\range\relax%
        \pgfmathtruncatemacro\result{ ( #1 >= \bt@rangea) && (#1 <= \bt@rangeb) }%
        \ifnum\result=1\relax%
            \breakforeach%
            \global\let\bt@inrange\@firstoftwo%
        \fi%
    }%
    \bt@inrange%
}
\newcommand\bt@getrangeb{%
    \@ifnextchar\relax%
        {\bt@rangeb=\bt@rangea}%
        {\@getrangeb}%
}
\def\@getrangeb-#1\relax{%
    \ifx\relax#1\relax%
        \bt@rangeb=100000%   \maxdimen is too large for pgfmath
    \else%
        \bt@rangeb=#1\relax%
    \fi%
}

%%%%%%%%%%%%%%%%%%%%%%%%%%%%%%%%%%%%%%%%%%%%%%%%%%%%%%%%%%%%%%%%%%%%%%%%%%%%%%
%
% \btLstHL<overlay spec>{range list}
%
% TODO BUG: \btLstHL commands can not yet be accumulated if more than one overlay spec match.
%
\newcommand<>{\btLstHL}[2]{%
  \only#3{\btIfInRange{\value{lstnumber}}{#1}{\color{#2}\def\lst@linebgrdcmd{\color@block}}{\def\lst@linebgrdcmd####1####2####3{}}}%
}%
\makeatother






% http://tex.stackexchange.com/questions/15237/highlight-text-in-code-listing-while-also-keeping-syntax-highlighting
%\usepackage[T1]{fontenc}
%\usepackage{listings,xcolor,beramono}
\usepackage{tikz}

\makeatletter
\newenvironment{btHighlight}[1][]
{\begingroup\tikzset{bt@Highlight@par/.style={#1}}\begin{lrbox}{\@tempboxa}}
{\end{lrbox}\bt@HL@box[bt@Highlight@par]{\@tempboxa}\endgroup}

\newcommand\btHL[1][]{%
  \begin{btHighlight}[#1]\bgroup\aftergroup\bt@HL@endenv%
}
\def\bt@HL@endenv{%
  \end{btHighlight}%
  \egroup
}
\newcommand{\bt@HL@box}[2][]{%
  \tikz[#1]{%
    \pgfpathrectangle{\pgfpoint{1pt}{0pt}}{\pgfpoint{\wd #2}{\ht #2}}%
    \pgfusepath{use as bounding box}%
    \node[anchor=base west, fill=orange!30,outer sep=0pt,inner xsep=1pt, inner ysep=0pt, rounded corners=3pt, minimum height=\ht\strutbox+1pt,#1]{\raisebox{1pt}{\strut}\strut\usebox{#2}};
  }%
}
\makeatother



\usetikzlibrary{calc}
\usepackage{xparse}%  For \NewDocumentCommand

% tikzmark command, for shading over items
\newcommand{\tikzmark}[1]{\tikz[overlay,remember picture] \node (#1) {};}

\makeatletter
\NewDocumentCommand{\DrawBox}{s O{}}{%
    \tikz[overlay,remember picture]{
    \IfBooleanTF{#1}{%
        \coordinate (RightPoint) at ($(left |- right)+(\linewidth-\labelsep-\labelwidth,0.0)$);
    }{%
        \coordinate (RightPoint) at (right.east);
    }%
    \draw[red,#2]
      ($(left)+(-0.2em,0.9em)$) rectangle
      ($(RightPoint)+(0.2em,-0.3em)$);}
}

\NewDocumentCommand{\DrawBoxWide}{s O{}}{%
    \tikz[overlay,remember picture]{
    \IfBooleanTF{#1}{%
        \coordinate (RightPoint) at ($(left |- right)+(\linewidth-\labelsep-\labelwidth,0.0)$);
    }{%
        \coordinate (RightPoint) at (right.east);
    }%
    \draw[red,#2]
      ($(left)+(-\labelwidth,0.9em)$) rectangle
      ($(RightPoint)+(0.2em,-0.3em)$);}
}

\NewDocumentCommand{\DrawBoxWideBlack}{s O{}}{%
    \tikz[overlay,remember picture]{
    \IfBooleanTF{#1}{%
        \coordinate (RightPoint) at ($(left |- right)+(\linewidth-\labelsep-\labelwidth,0.0)$);
    }{%
        \coordinate (RightPoint) at (right.east);
    }%
    \draw[black,#2]
      ($(left)+(-\labelwidth,0.9em)$) rectangle
      ($(RightPoint)+(0.2em,-0.3em)$);}
}
\makeatother

\usetikzlibrary{positioning}

\usetikzlibrary{shapes}

\hypersetup{
    colorlinks=true,
    linkcolor=blue,
    filecolor=magenta,
    urlcolor=cyan,
}



\usepackage{minted}

\begin{document}

\begin{frame}
	\titlepage
\end{frame}

\begin{frame}{Installation}
  \begin{itemize}
    \item Repository: \url{https://github.com/kokkos/kokkos-fortran-interop}
    \item Requirements:
      \begin{itemize}
        \item Kokkos 4.0 or newer
        \item C++17/Fortran08 compiler suites
      \end{itemize}
    \item Configure\\
      \texttt{cmake -DKokkos\_ROOT=/kokkos/path /interop/path}
  \end{itemize}
\end{frame}

\begin{frame}{What is Kokkos-Fortran-Interop?}
  \begin{itemize}
    \item Kokkos-Fortran offers wrappers around:
      \begin{itemize}
        \item \texttt{Kokkos::initialize(argc, argv)}
        \item \texttt{Kokkos::finalize()}
        \item \texttt{Kokkos::print\_configuration(output)}
        \item \texttt{Kokkos::View} 
        \item \texttt{Kokkos::DualView}
      \end{itemize}
    \item User kernels are written in C\texttt{++} $\rightarrow$ need to use
      \texttt{iso\_c\_binding}
    \item Only a subset of Kokkos capabilities are exposed
  \end{itemize}
\end{frame}

\begin{frame}{Starting a program}
  \begin{itemize}
    \item Similar to MPI, Kokkos needs to be initialized by calling
      \texttt{kokkos\_initialize} and finalized by calling \texttt{kokkos\_finalize}
    \item The \texttt{kokkos\_initialize} subroutine initializes Kokkos and
      reads command line arguments. It should be called after
      \texttt{MPI\_Initialize}
    \item \texttt{kokkos\_print\_configure("output.txt")} prints the
      Kokkos configuration to the file \texttt{output.txt}
  \end{itemize}
\end{frame}

\begin{frame}[containsverbatim]{Simple Example}
  \begin{minted}{fortran}
program my_kokkos_code
  use :: flcl_util_kokkos_mod

  ! Initialize Kokkos
  ! This subroutine reads command line arguments
  call kokkos_initialize()

  ! Print the configuration in a file
  call kokkos_print_configure('kokkos.out')

  ! Finalize Kokkos
  call kokkos_finalize()
end program my_kokkos_code
  \end{minted}
\end{frame}

\begin{frame}{Kokkos::View}
  \begin{itemize}
    \item \texttt{Kokkos::View} is Kokkos equivalent to an array
    \item \texttt{Kokkos::View} can have up to 8 dimensions in C++ but they are
      limited to 7 dimensions in Fortran due to limitation of the library
    \item Supported Fortran types: logical, 32-bit integer, 64-bit
      integer, 32-bit real, 64-bit real, 32-bit complex, 64-bit complex, and
      index (positive 64-bit integer)
    \item Types follow the pattern \texttt{view\_<type>\_<dimension>\_t}, e.g.,
      \texttt{view\_r64\_1d\_t} is a one-dimensional view of 64-bit real
  \end{itemize}
\end{frame}

\begin{frame}{Kokkos::View}
  \begin{itemize}
    \item The memory space defines where the memory is allocated
    \item Supported memory spaces: \texttt{Kokkos::HostSpace},
      \texttt{Kokkos::CudaManagedSpace}, and \texttt{Kokkos::HIPManagedSpace}
    \item The memory space is determined during the configuration of the
      Kokkos-Fortran-Interop library, based on the memory space
      configuration of Kokkos.
  \end{itemize}
\end{frame}

\begin{frame}{Kokkos::View}
  \begin{itemize}
    \item \texttt{Kokkos::View} can be allocated directly or built from an
      array. In the latter case, the \texttt{Kokkos::View} can only be used on
      the host % I don't see the point of this
    \item \texttt{Kokkos::View} that are allocated with
      \texttt{kokkos\_allocate\_view} must also be deallocated with
      \texttt{kokkos\_deallocate\_view}
    \item \texttt{kokkos\_allocate\_view} initializes all the elements of the
      \texttt{Kokkos::View} to zero
    \item \texttt{Kokkos::View} cannot be accessed directly from Fortran instead
      it can be accessed through a \texttt{pointer}
  \end{itemize}
\end{frame}

\begin{frame}[containsverbatim]{Kokkos::View}
  \begin{minted}{fortran}
use, intrinsic :: iso_c_binding
use, intrinsic :: iso_fortran_env
use :: flcl_mod

! Kokkos View only accessible from C++
type(view_r64_1d_t) :: v_c_y
! Pointer to access the Kokkos View from Fortran
real(real64), pointer, dimension(:) :: c_y
integer :: mm = 5000

call kokkos_allocate_view(c_y, v_c_y, 'c_y', &
                          int(mm, c_size_t))

! Do stuff

call kokkos_deallocate_view(c_y, v_c_y)
  \end{minted}
\end{frame}

\begin{frame}{Kokkos::DualView}
  \begin{itemize}
    \item \texttt{Kokkos::DualView} are similar to \texttt{Kokkos::View} but
      they are composed of two \texttt{Kokkos::View}s, one on the host and one on
      the device.
    \item It is the user's responsibility to synchronize the data
    \item The synchronization must be done in C++
    \item Supported memory spaces: \texttt{Kokkos::HostSpace},
      \texttt{Kokkos::Cuda}, \texttt{Kokkos::HIP}, and \texttt{Kokkos::SYCL}
    \item \texttt{Kokkos::DualView} cannot be accessed directly from Fortran instead
      it can be accessed through a \texttt{pointer}
  \end{itemize}
\end{frame}

\begin{frame}[containsverbatim]{Kokkos::DualView}
  \begin{minted}{fortran}
use, intrinsic :: iso_c_binding
use, intrinsic :: iso_fortran_env
use :: flcl_mod

real(real64), pointer, dimension(:) :: c_y
type(dualview_r64_1d_t) :: v_c_y
integer :: mm = 5000

call kokkos_allocate_dualview(c_y, v_c_y, 'c_y', &
                              int(mm, c_size_t))

! Do stuff

call kokkos_deallocate_dualview(c_y, v_c_y)
  \end{minted}
\end{frame}

\begin{frame}{Kernel}
  \begin{itemize}
    \item Kernels using Kokkos are written in C++
    \item Use C-binding from Fortran standard 
    \item Create a subroutine that calls the C++ function
  \end{itemize}
\end{frame}

\begin{frame}[containsverbatim]{Initialize View}
  \begin{minted}{fortran}
use, intrinsic :: iso_c_binding
use, intrinsic :: iso_fortran_env
use :: flcl_mod
use :: my_init_mod

real(real64), pointer, dimension(:) :: c_y
type(view_r64_1d_t) :: v_c_y
integer :: mm = 5000

call kokkos_allocate_view(c_y, v_c_y, 'c_y', &
                          int(mm, c_size_t))

call my_init(v_c_y)

call kokkos_deallocate_view(c_y, v_c_y)
  \end{minted}
\end{frame}

\begin{frame}[containsverbatim]{Initialize View}
  \begin{minted}{fortran}
module my_init_mod
    use, intrinsic :: iso_c_binding
    use, intrinsic :: iso_fortran_env
    use :: flcl_mod
    implicit none
    public
      interface
        subroutine my_f_init( y ) &
          & bind(c, name='my_c_init')
          import
          type(c_ptr), intent(in) :: y
        end subroutine my_f_init
      end interface
  \end{minted}
\end{frame}

\begin{frame}[containsverbatim]{Initialize View}
  \begin{minted}{fortran}
      contains

        subroutine my_init( y )
          type(view_r64_1d_t), intent(inout) :: y
          call my_f_init(y%ptr())
        end subroutine my_init
  
end module my_init_mod
  \end{minted}
\end{frame}

\begin{frame}[containsverbatim]{Initialize View}
  \begin{minted}{C++}
#include <Kokkos_Core.hpp>
#include <flcl-cxx.hpp>
using view_type = flcl::view_r64_1d_t;

extern "C" {
  void my_c_init(view_type **v_y) {
    view_type y = **v_y;
    Kokkos::parallel_for(
        "init", y.extent(0), 
        KOKKOS_LAMBDA(int idx) {y(idx) += idx;});
    Kokkos::fence();
  }
}
  \end{minted}
\end{frame}

\begin{frame}{Exercise}
  \begin{itemize}
    \item Use \texttt{Kokkos::View} to do an \emph{axpy}
    \item Do not forget to install the library
  \end{itemize}
\end{frame}

\end{document}
